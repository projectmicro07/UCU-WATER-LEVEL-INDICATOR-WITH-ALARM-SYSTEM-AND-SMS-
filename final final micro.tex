\documentclass[11pt]{article}
\usepackage[utf8]{inputenc}
\usepackage[affil-it]{authblk}
\usepackage{geometry}
\geometry{
 legalpaper,
 total={8in,13in},
 margin = 1in}

\usepackage{siunitx}
\usepackage{graphicx}
\usepackage{listings}
\usepackage{float}
\usepackage{subcaption}
\usepackage{latexsym}
\usepackage{textcomp}
\usepackage{color}

\title {Project in Microprocessor System \\Water Level Indicator with Alarm System and SMS Alert}

\author {THALIA MARIE C. CRISTOBAL \\ 

ANA FRANCISCA S. NANALI\\

ANGELO O. LASTRELLA \\

REYNANTE M. PASCUA  \\

CAREY MARK C. ILUIS \\

MARK E. BARROGA \\
 
GREATE M. SELGA \\

RICHARD MYRICK T. ARELLAGA (Instructor)}  


\affil{\textbf{Bachelor of Science in Computer Engineering} 
\\ 
\textbf{College of Engineering and Architecture}
\\
\textit{Urdaneta City University}}

\date{\emph{May 27, 2019}}

\begin{document}
\maketitle
\newpage
\pagenumbering{roman}
\tableofcontents                                                                   

\newpage
\pagenumbering{roman}

\section{\emph{Introduction}}

\paragraph{} {\large In this Arduino primarily based programmed water stage indicator and controller assignment the water degree is being estimated through utilizing ultrasonic sensors. The sensor used to measure water level is an ultrasonic sensor which is connected to Arduino. The ultrasonic sensor works on ECHO mechanism. The Ultrasonic Sensor conveys a high-recurrence sound heartbeat and after that occasions to what extent it takes for the reverberate of the sound to reflect. The sensor has 2 openings on its front. One opening transmits ultrasonic waves, (like a small speaker), alternate receives them, (like a minor amplifier). The ultrasonic sensor utilizes this information alongside the time difference amongst sending and getting the sound heartbeat to decide the separation to a question.}

\paragraph{} {\large In relation with the contemporary framework with so a good deal work and too much less time to spare, it is very tough to maintain in touch with the water level in the tanks. Water is vital in every hour of our lives. Hardly every person keeps in track of the level of water in the overhead tanks. The objective of the task is to measure the stage of water in the tank and notify the user about the water level via an SMS alert. This now not solely helps to maintain the tank full however also making it more convenient for our everyday chores and also heading off water wastage. In this project, the water is being measured by using ultrasonic sensors. Initially, the tank is regarded to be empty. When the sound waves are transmitted in the environment, they are mirrored back as ECHO. This identical concept is utilized this project. Waves generated by way of the ultrasonic sensors is dispatched to the water tank and their time of journeying and coming again is noted and after few calculations we can estimate the level of water in the tank. These alerts are sent as notifications in our phones via the GSM Module.}

\section{\emph{High Level Design}}
\paragraph{} { Our project on Water Level Indicator and Alarm System with SMS Alert format is to display the water level in different categories and their designated colors:

\paragraph{}- Normal water level {(green color)} 
\paragraph{}- Alert water level {(yellow color)} 
\paragraph{}- Alarm water level {(orange color)}
\paragraph{}- Critical water level {(red color)}

\paragraph{}  The GSM Module helps the user to be aware of the water degree in the tank through an SMS alert and additionally has trigger mechanism to get hold of SMS to turn OFF alarm when the system. It is very beneficial due to the fact the user want now not worry about the water content material at some stage in the peak hours of the day and keeps the user up to date concerning the water content.}

\section{\emph{Hardware Design}}
\subsection{\emph{Materials}} 

\textit{\Large Arduino UNO (ATmega 644P)}
\paragraph{} {\Large is a low-power CMOS 8-bit microcontroller based on the AVR enhanced RISC architecture. By executing powerful instructions in a single clock cycle, the ATmega644P achieves throughputs close to 1MIPS per MHz. This empowers system designer to optimize the device for power consumption versus processing speed.}

 \textit{\Large Ultrasonic Sensor (HC-SR04)}
\paragraph{} {\Large It is basically a distance sensor and is used for detecting the distance using SONAR method. It has two ultrasonic transmitters namely the receiver and the control circuit. The transmitter emits a high frequency ultrasonic sound wave which bounces off from any solid object and receiver receives it as an echo. 

\textit{\Large GSM Module SIM 900}
\paragraph{} {\Large It is widely used in mobile communication. It has a built in RS232 level converter. It has the ability to send SMS through SMS cell broadcast method.}

\textit {\Large Connecting Wires}
\paragraph{} {\Large In any electronic circuitry wires are the conductive connections between the elements in contact. Theoretically, they have zero resistance and provide perfect connections. On the breadboard, they look like nice colored jumper wires.}

\textit {\Large IC 7806}
\paragraph{} {\Large It is a voltage regulator integrated circuit. It belongs to the family of 78xx series of fixed linear voltage regulated ICs. The voltage source in a circuit may have fluctuations and would not give the fixed voltage as output. A constant output voltage value is maintained by this IC.}

\textit{\Large Buzzer}
\paragraph{} {\Large A device which served as the warning mechanism to the water level arises.}

 \textit {\Large Potentiometer}
\paragraph{} {\Large A device used to calibrate the distance efficiency of the ultrasonic sensor.}

 \textit {\Large 16X2 Liquid Crystal Display}
\paragraph{} {\Large A device that served as the display unit of the project to monitor the water level.}

\section{\emph{How it works?}}

The input ultrasonic sensor which detects the distance and the level of water then enters into the GizDuino with ATmega 644, the LCD will print what the ultrasonic sensor detect and will identify what level of water, the buzzer or piezo will rang and the GSM Module will be sent to people directly.


\section{\emph{Why did you choose?}}

\paragraph{} {\Large We named our project-study as Water Level Indicator With Alarm System and SMS Alert. We used the real situation in Rosales-Carmen Bridge as our field of study where in every situations that the water level and floods rises we observed that there is no alarming signals to reach the people or community along place where the bridge  is located so we thought of building a study that whenever the water level rises an SMS alert would be sent to people along the place. }

\paragraph{} {\Large Considering the depth of the tank, an estimation is drawn related to the distance size of the sensors to the base of the tank. The measurements have been finished using a tape measure and the results are tabulated as follows:}

\begin{table}[h!]
\begin{center}
\caption{Calculations}
\label{tab:table1}
\begin{tabular}{|l|l|l|l|}
      \hline
      \textbf{Level Warning} & \textbf{Actual Distance} & \textbf{Measured Distance} & \textbf{Percent Error}\\
      \hline Level 1 &40cm &30cm&0.1\\
      \hline Level 2 &40cm &30cm&0.5\\
      \hline Level 3 &40cm &30cm&0.2\\
      \hline Level 4 &40cm &30cm&0.3\\ 
      \hline
      
    \end{tabular}
  \end{center}
\end{table}

\section{\emph{Software Design}}

\subsection{\emph{Peripherals of the MCU}}\

LED connection pins (13,12,11,10),
Buzzer pin 14, and
Liquid Crystal LCD connection pins (7,6,5,4,3,2)

\subsection{\emph{Algorithms}}

\begin{lstlisting}

#define buzzer 14
#define led 31
#define led1 30
#define led2 29
#define led3 28
#include <LiquidCrystal.h>

int modem_response;
int PWON= A5;
char aux_string[30];
char* owner = "09272358040";

int sound = 250;
LiquidCrystal lcd(9,8,7,6,5,4);

const int trigPin = 15;
const int echoPin = 16;
const char lock1 = 2;
const char lock2 = 3;


const int gsmTimeout = 30;
const int gsmMaxBuffer = 128;
char gsmBuffer[gsmMaxBuffer] = "";

void setup() {

  pinMode(lock1,OUTPUT);
  pinMode(lock2,OUTPUT);
  pinMode(trigPin, OUTPUT);
  pinMode(echoPin, INPUT);
  pinMode(buzzer, OUTPUT);
  pinMode(led, OUTPUT);
  pinMode(led1, OUTPUT);
  pinMode(led2, OUTPUT);
  pinMode(led3, OUTPUT);

  lcd.begin (16,2);
  lcd.print ("H2OLvL indicator");
  lcd.setCursor(0,1);
  lcd.print ("Project MICRO");
  delay(100);
  
    Serial1.begin(9600);
    Serial.begin(9600);    

    AUTO_PWON();
    Connect();
    delay(100);
    FireUpModem();
  
}
void loop() {

  long duration, distance;
  digitalWrite(trigPin, LOW); 
  delayMicroseconds(2);
  digitalWrite(trigPin, HIGH);
  delayMicroseconds(10);
  digitalWrite(trigPin, LOW);
  duration = pulseIn(echoPin, HIGH);
  distance = (duration/2) / 29.1;

  Serial.print("DISTANCE = ");
  Serial.println(distance);

  CheckForMessages();
  
if (distance > 40 || distance <= 30){
    digitalWrite(led, LOW);
  }
  else {
    lcd.setCursor(0,0);
    lcd.print("Warning level 1 ");
    lcd.setCursor(0,1);
    lcd.print("WaterLvL(cm): ");
    lcd.print(distance);
    delay(10);
    digitalWrite(led, HIGH);
  }
  if (distance > 30 || distance <= 20){
    digitalWrite(led1, LOW);
  }
  else {
    lcd.setCursor(0,0);
    lcd.print("Warning level 2 ");
    lcd.setCursor(0,1);
    lcd.print("WaterLvL(cm): ");
    lcd.print(distance);
    delay(10);
    digitalWrite(led1, HIGH);
  }
  if (distance > 20 || distance <= 9){
    digitalWrite(led2, LOW);
    noTone(buzzer);
  }
  else {
    lcd.setCursor(0,0);
    lcd.print("Warning level 3 ");
    lcd.setCursor(0,1);
    lcd.print("WaterLvL(cm): ");
    lcd.print(distance);
    delay(10);
    tone(buzzer, sound);
    delay(50);
    digitalWrite(led2, HIGH);
  }
  if (distance > 9 || distance <= 0){
    digitalWrite(led3, LOW);
    noTone(buzzer); 
   
      
  }
  else {
    lcd.setCursor(0,0);
    lcd.print("Warning level 4 ");
    lcd.setCursor(0,1);
    lcd.print(" WaterLvL(cm): ");
    lcd.print(distance);
    delay(10);
    tone(buzzer, sound);
    digitalWrite(led3, HIGH); 
    SendMessage("WATER LEVEL WARNING LEVEL4. FORCE EVACUATION!",owner);
    delay(100);
   
  }
     delay(100);
}

void CheckForMessages()
{
   if(Serial1.available()){
     byte charsRead = Serial1.readBytesUntil('\n',gsmBuffer,gsmMaxBuffer);
     if(charsRead){
       gsmBuffer[charsRead] = 0; //Terminate String
       _gsmSerialHandleLine(String(gsmBuffer));
       
     }
   }
}


boolean _isNewSms(const String &line)
{
  // +CMTI: "SM",11
  return line.indexOf("+CMTI") == 0 &&
         line.length() > 12;
}

boolean _isSms(const String &line)
{
  return line.indexOf("+CMGR") == 0;
}

void _receiveTextMessage(const String &line)
{
  String index = line.substring(12);
  index.trim();
  Serial1.flush();
  Serial1.print("AT+CMGR=" + index + "\r");
}

boolean _gsmReadBytesOrDisplayError(char numberOfBytes)
{
  if(!_gsmWaitForBytes(numberOfBytes, gsmTimeout))
  {
     Serial.println("Error Reading From Device");
    return false;
  }

  while(Serial1.available())
  {
    char next = Serial1.read();
   
  }
  
  return true;
}

char _gsmWaitForBytes(char numberOfBytes, int timeout)
{

  while(Serial1.available() < numberOfBytes)
  {
    delay(200);
    timeout -= 1;
    if(timeout == 0)
    {
      return 0;
    }
  }
  return 1;
}

boolean _readAndPrintSms()
{
  // Line contains the +CMGR response, now we need to read the message
  
  byte charsRead = Serial1.readBytesUntil('\n', gsmBuffer, gsmMaxBuffer);
  
  if(charsRead > 1)
  {
    charsRead--; // Skip newline
    gsmBuffer[charsRead] = 0; // Terminate string
    _gsmReadBytesOrDisplayError(4); // OK\r\n
    Serial.println(gsmBuffer);
    
  }
}

void _gsmSerialHandleLine(const String &s)
{
  if(_isNewSms(s))
  {
      _receiveTextMessage(s);
  }
  else if(_isSms(s))
  {
    _readAndPrintSms();
  }
  else
  {
    
  }
}

int SendMessage(char* message, char* number)
{
  sprintf(aux_string,"AT+CMGS=\"%s\"", number);
  modem_response = SendModemCommand(aux_string, ">", 2000); 
  if(modem_response == 1)
   {
     Serial1.print(message);
     Serial1.write(0x1A);
     modem_response = SendModemCommand("","OK",20000);
     if(modem_response == 1)
      {
        Serial.println("Message Sent");
      }else
      {
        Serial.println("Message not Sent");
      }
   } 
}

void AUTO_PWON(){

    int answer=0;
    
    answer = SendModemCommand("AT", "OK", 2000);
    if (answer == 0)
    {
       
        digitalWrite(PWON,HIGH);
        delay(3000);
        digitalWrite(PWON,LOW);
    
       
        while(answer == 0){   
            answer = SendModemCommand("AT", "OK", 2000);    
        }
    }
    
}

void FireUpModem(){
  
  Serial.println("Initializing Modem");
  SendModemCommand("AT+CMGDA=\"DEL ALL\"","OK",10000);
  SendModemCommand("AT+CNMI=3,1","OK",1000);
  SendModemCommand("AT+CMGF=1","OK",1000);
  Serial.println("Modem Initialization OK");
}

void Connect()
{
    Serial.println("Connecting to nearest cell site");

    while( (SendModemCommand("AT+CREG?", "+CREG: 0,1", 500) || SendModemCommand("AT+CREG?", "+CREG: 0,5", 500)) == 0 ); 
    Serial.println("Connected");
}


int SendModemCommand(char* cmd, char* resp, unsigned int tout)
{
    int ctr=0;
    int ans=0;
    char buff[100];
    unsigned long prev;

    memset(buff, '\0', 100);
    
    delay(100);
    
    while( Serial1.available() > 0) Serial1.read(); 
    
    Serial1.println(cmd);    


    ctr = 0;
    prev = millis();


    do{

        if(Serial1.available() != 0){    
            buff[ctr] = Serial1.read();
            ctr++;
             if (strstr(buff, resp) != NULL)    
            {
                ans = 1;
            }
        }

    }while((ans == 0) && ((millis() - prev) < tout));    

    return ans;
}
\end{lstlisting}

\subsection{\emph{State Design}}

\begin{figure}[H] 
\centering
\includegraphics[width=0.5\textwidth]{assets/21.png}
\caption{}
  
\end{figure} 

\subsection{\emph{Flow Charts}}

\begin{figure}[h!]
\centering
\includegraphics[width=0.5\textwidth]{assets/20.png}
\caption{Flow Charts}

\end{figure}

\section{\emph{Results}} 
When the ultrasonic sensor detects the distance of water level, the buzzer will alarm and send an SMS automatically to inform the people or community.

\section{\emph{Conclusions}}

This is a simple model of water level indicator with alarm system and sms alert. This project is a solution to reached people or community along the place where the bridge is located, so we thought of building this project that whenever the water level rises an sms alert would be sent.

After all the hardwork, we realized and need to be more patience and willing to sacrifices in order to create a useful or desirable output to prove that our project is going to function well according to the plan.

\section{\emph{Appendix}}

\subsection{Cost Details}\

\begin{table}[h!]
\begin{center}
\caption{Table}
\label{tab:table1}
\begin{tabular}{|l|l|l|l|}

\hline
              \textbf{Product} & \textbf{Quantity} & \textbf{Price} & \textbf{Total}\\

      \hline {\small HC-SR04 Distance Sensor}                           &1  &60.00    &60.00\\
      \hline {\small High Quality Passive Buzzer Module}	            &1	&27.00	   &27.00\\
      \hline {\small Male to Female 5-Pins Connecting Wire 40cm}	    &4	&21.50	   &86.00\\
      \hline {\small Male to Male 5-Pins Connecting Wire 40cm}	        &4	&20.00	   &80.00\\ 
      \hline {\small Kinsten Presensitized PCB (Single Sided 4x3)}    	&1	&50.00	   &50.00\\
      \hline {\small gizDuino + with ATmega 644}	                    &1	&555.00   &555.00\\
      \hline {\small Sim900A GSM Module with Antenna and Power Cable}	&1	&1,000.00 &1,000.00\\
      \hline {\small SRA 09VDC-CL Relay}                                &1	&41.00	   &41.00\\ 
      \hline {\small LCD Module 2x16 Blue}                              &1	&115.00   &115.00\\
      \hline {\small 10K Trimmer Potentiometer}                         &3	&15.00	   &45.00\\
      \hline {\small LM7806}                                            &2	&19.50.00 &39.00\\
      \hline {\small Shipping}                                          &   &505.50   &505.50\\ 
      \hline {\small Foods}                                             &   &620.00   &620.00\\
      \hline {\small piezo}                                             &1   &80.00    &80.00\\
      \hline {\small sim card}                                         &1    &40.00    &40.00\\
       \hline {\small LED}                                              &1 &24.00 &24.00 \\
       \hline {\small TOTAL}                                             &   &.00      &.00\\
         \hline
      
    \end{tabular}
  \end{center}
\end{table}


\subsection{\emph{PCB Designs}}\

\begin{figure}[H] 
\centering
\includegraphics[width=0.5\textwidth]{assets/44.jpg}
\caption{}
\end{figure}
  

\subsection{\emph{Full Schematics}}\

\begin{figure}[H] 
\centering
\includegraphics[width=0.5\textwidth]{assets/26.jpg}
\caption{Actual Image}
\end{figure}

\subsection{\emph{Task Breakdown}}\

{\emph{Hardware}
\begin{enumerate}
\item Angelo O. Lastrella 
\item  Mark E. Barroga 
\item Thalia Marie C. Cristobal 
\item Reynante M. Pascua 
\item Ana Francisca S. Nanali 
\item Greate M. Selga 
\item Carey Mark C. Iluis 
\end{enumerate}

{\emph{Tech Writer}
\begin{enumerate}
\item Angelo O. Lastrella 
\item Mark E. Barroga 
\item Thalia Marie C. Cristobal 
\item Reynante M. Pascua 
\item Ana Francisca S. Nanali 
\end{enumerate}

\subsection{\emph{References}}\
\begin{enumerate}
\item Mali, M., Satish, S., Satish, S. and Satish, K. (2019). Live Water Level Indicator with SMS and Voice Call Alerts using Arduino and Ultrasonic Sensor. [ebook] Available at: http://ijcset.net/docs/Volumes/Volume%209/ijcset2019090102.pdf [Accessed 2019]. 

\end{enumerate}

\subsection{\emph{Source code reference}}\
\begin{enumerate}

\item https://maker.pro/arduino/projects/ultrasonic-arduino-water-level-indicator
\item https://mechatrofice.com/arduino/ultrasonic/water-level-indicator
\item https://hub360.com.ng/automatic-water-level-indicator-and-controller-using-ultrasonic-sensor-hc-sr04/

\end{enumerate}
\end{document}